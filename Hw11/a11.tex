\documentclass[12pt,letterpaper,boxed,cm]{hmcpset}

\usepackage[margin=1in]{geometry}
\usepackage{mathtools}
\usepackage{mathrsfs}
\usepackage{graphicx}
\usepackage{cases}
\usepackage{enumitem}
\usepackage{wasysym}

\name{A GIRL HAS NO NAME}
\class{Computer Science 81}
\assignment{Homework 11}
\duedate{4/18/17}

\newcommand{\pn}[1]{\left( #1 \right)}
\newcommand{\abs}[1]{\left| #1 \right|}
\newcommand{\bk}[1]{\left[ #1 \right]}
\newcommand{\set}[1]{\left\{#1\right\}}

\begin{document}
\problemlist{1, 2, 3, 4, 5, 6}

\begin{problem}[1.]
    [2 points] Consider the language 
    \[
        \mathbf{Reach} = \set{<M, x, s> | M~\text{eventually reaches control state $s$ when started on input $x$}}. 
    \]
    Is \textbf{Reach} recognizable, co-recognizable, both, or neither? Justify your answers.
\end{problem}

\begin{solution}
    \vfill
\end{solution}
\newpage

\begin{problem}[2.]
    [6 points] Define a language to have ``prime nature'' iff every string in it has a prime number of symbols. In other words, the language is a subset of 
    \[
        \mathbf{Primes} = \set{1^p~|~p\text{ is prime}} = \set{11, 111, 11111, 1111111, \ldots}
    \]
    Any language containing a string not of prime length, such as the language $\set{111, 1111}$ does not have prime nature. Define language \textbf{PN} as follows:
    \[
        \textbf{PN} = \set{ <M> |~M\text{ is a Turing machine and $L(M)$ has prime nature} }
    \]
    Show that \textbf{PN} is undecidable. Note that whether \textbf{PN} itself has prime nature is not
    at issue. (It is unlikely that it would.)
\end{problem}

\begin{solution}
    \vfill
\end{solution}
\newpage

\begin{problem}[3.]
    [2 points] In the previous problem, is \textbf{PN} recognizable? Is \textbf{PN} co-recognizable? Justify.
\end{problem}

\begin{solution}
    \vfill
\end{solution}
\newpage

\begin{problem}[4.]
    [2 points] Show that if $L$ is a recognizable language, and $L \le_m L^c$ (i.e. $L$ is \emph{mapping reducible} to its own complement) then $L$ is decidable.
\end{problem}

\begin{solution}
    \vfill
\end{solution}
\newpage

\begin{problem}[5.]
    [10 points] Show that 
    \[
        \textbf{Infinite}_{\text{TM}} = \set{<M> |~M \text{ is a Turing machine and $L(M)$ is infinite}}
    \]
    is neither recognizable nor co-recognizable. (This requires two separate proofs, using different techniques described in the lecture slides.)
\end{problem}

\begin{solution}
    \vfill
\end{solution}
\newpage

\begin{problem}[6.]
    [8 points] For each of the following questions for arbitrary Turing machine codes $<M>$, is the question rendered unrecognizable or un-corecongizable by Rice's theorem?  If so, state which (unrecognizable or un-corecongizable). If Rice's theorem doesn't apply, so state and indicate whether the question is decidable, giving the best justification that you can. You may use the Church-Turing thesis in describing decidable cases.
    \begin{enumerate}[label=\alph*.]
        \item $M$ accepts more than 81 different strings.
        \item $M$ has no rejecting control states.
        \item $M$ uses more than 81 steps on some input without halting.
        \item $L(M)$ is decidable by some finite-state machine.
    \end{enumerate}
\end{problem}

\begin{solution}
    \vfill
\end{solution}
\newpage

\end{document}
